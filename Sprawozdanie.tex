\documentclass[11pt,a4paper]{article}
\setlength{\parindent}{10pt}
\usepackage{polski}
\usepackage[utf8]{inputenc} 
\usepackage{graphicx}
\usepackage{hyperref}

\title{MODI - sprawozdanie z pracy domowej}

\author{Jakub Sikora, 283418}
\date{\today}


\begin{document}
\maketitle
\begin{center}
Politechnika Warszawska\linebreak 
Wydział Elektroniki i Technik Informacyjnych
\end{center}
\newpage
\tableofcontents
\newpage

\section{Wstęp}\label{sec:Wstęp}
\indent \indent Tematem pracy domowej były charakterystyki statyczne nieliniowe modeli o 1 wejściu i jednym wyjściu (model SISO - Single Input Single Output) a także o 2 wejściach i 1 wyjściu (model MISO - Multiple Input Single Output). Dodatkowo, celem zadania było również zlinearyzowanie danych charakterystyk w dowolnych punktach pracy i porównanie wyników eksperymentu.

\section{Charakterystyki nieliniowe}\label{sec:Charakterystki nieliniowe}
\subsection{Modele nieliniowe}
\indent \indent Model statyczny procesu o jednym wejściu $u$ oraz jednym wyjściu $y$ dany jest wzorem:

\begin{equation}
    y(u) = cos^2(0,25u) - 0,01u^3 - 0,02
    \label{eqn:wzor1}
\end{equation}


Silną nieliniowość wnoszą tutaj funkcje $(cos(0.25u))^2$ oraz $0.01u^3$. Model ten będę badać w zakresie $-5 \le u \le 5$.
\newline 

\indent Model statyczny procesu o dwóch wejściach $u_{1}$ i $u_{2}$ oraz jednym wyjściu dany jest wzorem:

\begin{equation}
	y(u_1, u_2) = -10u_{1} + 5u_{2}^3
	\label{eqn:wzor2}
\end{equation}

Elementem wnoszącym nieliniowość jest wyrażenie $5u_2^3$. Ten model będę badać w zakresach $-1 \le u_{1} \le 1$ oraz $ -1 \le u_{2} \le 1 $. 

\subsection{Wykresy charakterystyk}
\indent \indent Za pomocą środowiska \texttt{Matlab}, wykreśliłem charakterystyki obu modelu w odpowiednich zakresach. Kod generujący wykresy obu charakterystyk 
zawarty jest w skrypcie \texttt{nonlinear\_figures.m}. Efekt pracy znajduje się na następnej stronie. 

\newpage
\subsubsection{Model SISO}

\begin{figure}[h]
\includegraphics[width=1\textwidth]{SISO_nonlinear}
\caption{Charakterystyka modelu SISO}
\end{figure}

Niebieską linią zaznaczono charakterystykę modelu w zakresie $-5 \le u \le 5$. W danym przedziale funkcja ma dwa ekstrema w punktach $u = 0$ oraz $u \approx -2.8738$, które wpływają na nieliniowość funkcji. Dodatkowo, funkcja szybko maleje w przedziale $[2,5]$ (moduł pochodnej jest większy niż w innych miejscach).

\newpage
\subsubsection{Model MISO}

\begin{figure}[h]
\includegraphics[width=1\textwidth]{MISO_nonlinear}
\caption{Charakterystyka modelu MISO}
\end{figure}

Na wykresie nakreślono charakterystykę modelu w zakresie \\$-1 \le u_{1},u_{2} \le 1$. Model przy zmianie sterowania $u_{1}$ zachowuje się liniowo. Wynika to z braku elementów nieliniowych w funkcji modelu zależnych od $u_{1}$. Przy zmianie $u_{2}$ odpowiedź ma już charakter nieliniowy związany z funkcją $5u_{2}^3$. W zadanych przedziałach funkcja osiąga największą wartość dla $(u_{1},u_{2}) = (-1,1)$ równą $15$.

\newpage

\section{Linearyzacja charakterystyk}\label{sec:Linearyzacja charakterystyk}

\indent \indent W celu umożliwienia dokładniejszej analizy procesu danego modelem nieliniowym, taki model poddaje się linearyzacji w punkcie pracy. Przybliżenie funkcji modelu, funkcją liniową pozwala na zastosowanie dobrze sprawdzonych metod analizy i projektowania. W przypadku słabo nieliniowych funkcji, metoda ta ma dużą dokładność. Niestety, metoda ta traci na dokładności gdy funkcja jest silnie nieliniowa. W takim przypadku najlepiej podzielić interesujący nas przedział pracy na kilka podprzedziałów i każdemu przypisać odpowiednią funkcję liniową. Na wykresie, charakterystyka liniowa jest po prostu styczną do wykresu charakterystyki nieliniowej.

\subsection{Linearyzacja funkcji wielu zmiennych}
\indent W ogólności wzór na linearyzację w otoczeniu punktu pracy $\bar{u} = (\bar{u_{1}}, ... , \bar{u_{n}})$: 

\begin{equation}
	y(u_{1}, .., u_{n}) \approx y(\bar{u}) + \frac{\partial y(\bar{u})}{\partial u_{1}}(u_{1} - \bar{u_{1}}) + ... + \frac{\partial y(\bar{u})}{\partial u_{n}}(u_{n} - \bar{u_{n}})
	\label{eqn:wzor3}
\end{equation}

W przypadku funkcji jednej zmiennej wzór sprowadza się do postaci: 
\begin{equation}
	y(u) \approx y(\bar{u}) + \frac{dy(\bar{u})}{du}(u - \bar{u})
	\label{eqn:wzor4}
\end{equation}

W przypadku modelu z dwoma wejściami i jednym wyjściem wzór przyjmuje formę:
\begin{equation}
	y(u_{1}, u_{2}) \approx y(\bar{u_{1}}, \bar{u_{2}}) + \frac{\partial y(\bar{u_{1}}, \bar{u_{2}})}{\partial u_{1}}(u_{1} - \bar{u_{1}}) +  \frac{\partial y(\bar{u_{1}}, \bar{u_{2}})}{\partial u_{2}}(u_{2} - \bar{u_{2}})
	\label{eqn:wzor5}
\end{equation}

\subsection{Linearyzacja modelu SISO}

W celu wyznaczenia modelu zlinearyzowanego należy policzyć pochodną funkcji $y(u)$. 
\begin{eqnarray*}
\indent \frac{dy(u)}{du} = ((cos(0,25u))^2 - 0,01u^3 - 0,02)'
\end{eqnarray*}
Rozbijam pochodną sumy na sumę pochodnych: 
\begin{eqnarray*}
\indent \frac{dy(u)}{du} = (cos^2(0,25u))' - (0,01u^3)' - (0,02)' 
\end{eqnarray*}
Obliczam poszczególne pochodne ze wzoru na pochodną funkcji złożonej:
\begin{eqnarray*}
\indent \frac{dy(u)}{du} = \frac{1}{2}cos(0,25u)(-sin(0,25u) - 0,03u^2  
\end{eqnarray*}
Wyrażenie z funkcjami trygonometrycznymi sprowadzam do uproszczonej postaci, korzystając ze wzoru na sinus podwojonego kąta:
\begin{eqnarray*}
\indent \frac{dy(u)}{du} = -\frac{1}{4}sin(0,5u) - 0,03u^2  
\end{eqnarray*}

\indent Ostateczna postać zlinearyzowanej funkcji modelu ma postać:
\begin{equation}
y_{lin}(u) \approx cos^2(0,25\bar{u}) - 0,01\bar{u}^3 - 0,02 + [-\frac{1}{4}sin(0,5\bar{u}) - 0,03\bar{u}^2](u - \bar{u})
\label{eqn:wzor6}
\end{equation}

\subsection{Linearyzacja modelu MISO}
W celu wyznaczenia modelu zlinearyzowanego należy policzyć pochodne cząstkowe funkcji $y(u_{1}, u_{2})$ czyli $\frac{\partial y(u_{1}, u_{2})}{\partial{u_{1}}}$ i $\frac{\partial y(u_{1}, u_{2})}{\partial{u_{2}}}$.
\\ \\
Obliczam pochodną cząstkową po $\partial u_{1}$:
\begin{eqnarray*}
\indent \frac{\partial y(u_{1}, u_{2})}{\partial{u_{1}}} = \frac{\partial}{\partial u_{1}}(-10u_{1} + 5u_{2}^3)
\end{eqnarray*}

\begin{eqnarray*}
\indent \frac{\partial y(u_{1}, u_{2})}{\partial{u_{1}}} = -10
\end{eqnarray*}
\\
Obliczam pochodną cząstkową po $\partial u_{2}$:
\begin{eqnarray*}
\indent \frac{\partial y(u_{1}, u_{2})}{\partial{u_{2}}} = \frac{\partial}{\partial u_{2}}(-10u_{1} + 5u_{2}^3)
\end{eqnarray*}

\begin{eqnarray*}
\indent \frac{\partial y(u_{1}, u_{2})}{\partial{u_{2}}} = 15u_{2}^2
\end{eqnarray*}
\\
Ostateczna postać zlinearyzowanej funkcji modelu:
\begin{equation}
y_{lin}(u) \approx -10\bar{u_{1}} + 5\bar{u_{2}}^3 - 10(u_{1} - \bar{u_{1}}) + 15\bar{u_{2}}(u_{2} - \bar{u_{2}})
\label{eqn:wzor7}
\end{equation}
\newpage

\section{Charakterystyki zlinearyzowane a punkty pracy}
\indent \indent Charakterystki zlinearyzowane obliczyłem ponownie za pomocą środowiska \texttt{Matlab}. Skrypt odpowiedzialny za kreślenie charakterystyk modelu o jednym wejściu nazywa się \texttt{SISO\_linear\_figures.m}, natomiast skrypt zajmujący się kreśleniem charakterystyk modelu o dwóch wejściach ma nazwę \texttt{MISO\_linear\_figures.m}.

\subsection{Linearyzacje modelu SISO} 
\indent \indent W zależności od doboru punktu pracy, charakterystyki zlinearyzowane różnią się nachyleniem. W ogólności punkt pracy wybierany jest tak aby obiekt opisywany modelem, pracował optymalnie. W przypadku tego zadania, dobrałem trzy punkty pracy tak, aby nachylenia prostych charakterystyk różniły się jak najbardziej.
\begin{center}
Dobrane punkty pracy: $\bar{u_{1}} = -3, \bar{u_{2}} = -1, \bar{u_{3}} = 3$
\end{center}
\flushleft

\subsubsection{Punkt pracy $\bar{u} = -3$}

\begin{figure}[h]
\includegraphics[width=0.9\textwidth]{SISO_linear1}
\caption{Charakterystyka zlinearyzowana modelu SISO w punkcie pracy $\bar{u} = -3$}
\end{figure}
\indent \indent Przy doborze punktu pracy blisko ekstremum lokalnego, charakterystyka zlinearyzowana przypomina funkcję stałą. Taka linearyzacja bardzo słabo linearyzuje model na zadanym przedziale. 
\newpage

\subsubsection{Punkt pracy $\bar{u} = -1$}

\begin{figure}[h]
\includegraphics[width=0.9\textwidth]{SISO_linear2}
\caption{Charakterystyka zlinearyzowana modelu SISO w punkcie pracy $\bar{u} = -1$}
\end{figure}

\indent \indent Dobór punktu pracy $\bar{u} = -1$ dobrze przybliża oryginalną funkcję w przedziale $[-3,0]$, jednak w pozostałych fragmentach znacznie odbiega od oryginalnej funkcji. Warto zwrócić uwagę że funkcja liniowa jest na całym przedziale rosnąca, choć funkcja nieliniowa w ogólności jest malejąca, oprócz zakresu między ekstremami $[-2,8738;0]$.
\newpage

\subsubsection{Punkt pracy $\bar{u} = 3$}

\begin{figure}[h]
\includegraphics[width=0.9\textwidth]{SISO_linear3}
\caption{Charakterystyka zlinearyzowana modelu SISO w punkcie pracy $\bar{u} = 3$}
\end{figure}

\indent \indent Ostatnim wybranym przeze mnie punktem pracy jest punkt $\bar{u} = 3$. Taka linearyzacja dobrze przybliża charakterystykę nieliniową w zakresie $[1,5]$. Z powodu ujemnej wartości pochodnej w punkcie $\bar{u} = 3$, charakterystyka liniowa szybko maleje wraz ze wzrostem argumentu. Tempo malenia funkcji liniowej jest większe niż funkcji oryginalnej, czego wynikiem jest duża rozbieżność wartości charakterystyk w lewej części wykresu.
\newpage
   
\subsection{Linearyzacje modelu MISO}
\indent \indent W przypadku dwuwymiarowym punkty pracy są w ogólności wektorami a  charakterystyki zlinearyzowane z stycznych, stają się płaszczyznami. W przypadku tego zadania, dobrałem trzy punkty pracy tak, aby nachylenia tych płaszczyzn różniły się jak najbardziej oraz aby ukazać różnice między charakterystykami.
\begin{center}
Dobrane punkty pracy: $(\bar{u_{1}},\bar{u_{2}}) = (0,0); (\bar{u_{1}},\bar{u_{2}}) = (1,-1); (\bar{u_{1}},\bar{u_{2}}) = (-1,-1)$.
\end{center}
\flushleft

\subsubsection{Punkt pracy $(\bar{u_{1}},\bar{u_{2}}) = (0,0)$}

\begin{figure}[h]
\includegraphics[width=0.9\textwidth]{MISO_linear1}
\caption{Charakterystyka zlinearyzowana modelu MISO w punkcie pracy $(\bar{u_{1}},\bar{u_{2}}) = (0,0)$.}
\end{figure}

\indent \indent W przypadku doboru punktu pracy w ekstremum funkcji nieliniowej, otrzymana płaszczyzna bardzo dobrze przybliża oryginalny nieliniowy model. Dodatkowo w osi $u_{1}$, charakterystyka zlinearyzowana w dużej mierze pokrywa się z modelem nieliniowym. Wynika to z liniowego charakteru funkcji zależnej od $u_{1}$. 
\newpage

\subsubsection{Punkt pracy $(\bar{u_{1}},\bar{u_{2}}) = (1,-1)$}

\begin{figure}[h]
\includegraphics[width=0.9\textwidth]{MISO_linear2}
\caption{Charakterystyka zlinearyzowana modelu MISO w punkcie pracy $(\bar{u_{1}},\bar{u_{2}}) = (1,-1)$.}
\end{figure}

\indent \indent W przypadku punktu pracy równego $(\bar{u_{1}},\bar{u_{2}}) = (1,-1)$, uzyskana charakterystka liniowa mocno odbiega od oryginalnej charakterystki nieliniowej. Mocno widoczna jest rozbieżność najwyższych wartości. Dla funkcji nieliniowej jest ona równa $15$, a dla funkcji liniowej wynosi aż $35$.  
\newpage

\subsubsection{Punkt pracy $(\bar{u_{1}},\bar{u_{2}}) = (1,1)$}

\begin{figure}[h]
\includegraphics[width=0.9\textwidth]{MISO_linear3}
\caption{Charakterystyka zlinearyzowana modelu MISO w punkcie pracy $(\bar{u_{1}},\bar{u_{2}}) = (1,1)$.}
\end{figure}

\indent \indent W przypadku punktu pracy znajdującego się po drugiej stronie osi $OY$ wykresu $(\bar{u_{1}},\bar{u_{2}}) = (1,1)$, obserwowalna jest symetryczna sytuacja w stosunku do poprzedniego punktu pracy. Tym razem wartości największe są zgodne, jednak tym razem wartości minimalne znacznie odbiegają od siebie.
\newpage

\section{Wnioski}
\indent Linearyzacja funkcji nieliniowych jest bardzo przydatnym narzędziem w analizie modeli, jest również niezbędna przy projektowaniu układów regulacji automatycznej. Należy przy tym pamiętać że linearyzacja funkcji nieliniowej wiąże się z utratą dokładności, dlatego warto wziąć pod uwagę że linearyzacja jest operacją lokalną i że można podzielić docelowy przedział na kilka podprzedziałów i tam dokonać lokalnego przybliżenia funkcją liniową.\\
\ 
\section{Dodatkowe informacje}
\flushleft{ \indent Wszystkie obliczenia i wykresy zostały zrealizowane za pomocą pakietu \texttt{Matlab}. Do sprawozdania dołączam wszystkie pliki źródłowe, z dokładnymi opisami w postaci komentarzy. Aby zrealizować wszystkie obliczenia po raz kolejny należy uruchomić skrypt \texttt{execute.m} i postępować zgodnie z poleceniami wypisywanymi na konsoli. Skrypt wywoła odpowiednie funkcje obliczające charakterystyki nieliniowe i liniowe, wygeneruje wykresy dla podanych wcześniej punktów pracy i zapisze je do folderu \texttt{figures}.}


\end{document}
